\documentclass[a4paper,UTF8]{article}

\usepackage[margin=1.25in]{geometry}
\usepackage{color}
\usepackage{graphicx}
\usepackage{amssymb}
\usepackage{amsmath}
\usepackage{amsthm}
\usepackage{enumerate}
\usepackage{bm}
\usepackage{hyperref}
\usepackage{epsfig}
\usepackage{color}
\usepackage{mdframed}
\usepackage{lipsum}
\usepackage{mathtools}
\usepackage{algorithm}
\usepackage{algorithmic}
\usepackage{listings}
\usepackage{xcolor}
\usepackage{float}
\usepackage{caption}

\usepackage[utf8]{inputenc}
\usepackage[UTF8]{ctex}

\newmdtheoremenv{thm-box}{myThm}
\newmdtheoremenv{prop-box}{Proposition}
\newmdtheoremenv{def-box}{define}

\setlength{\evensidemargin}{.25in}
\setlength{\textwidth}{6in}
\setlength{\topmargin}{-0.5in}
\setlength{\topmargin}{-0.5in}

\usepackage{indentfirst}
\setlength{\parindent}{2em}

\usepackage{subfigure}
% \setlength{\textheight}{9.5in}
%%%%%%%%%%%%%%%%%%set header and footer here%%%%%%%%%%%%%%%%%%
\usepackage{fancyhdr}
\usepackage{lastpage}
\usepackage{layout}
\footskip = 10pt
\pagestyle{fancy}
\lhead{2020, Spring}
\chead{大数据综合处理实验}
\rhead{实验四}
\cfoot{\thepage}
\renewcommand{\headrulewidth}{1pt}  			%header
\setlength{\skip\footins}{0.5cm}    			
\renewcommand{\footrulewidth}{0pt}  		

\makeatletter 							
\def\headrule{{\if@fancyplain\let\headrulewidth\plainheadrulewidth\fi%
\hrule\@height 1.0pt \@width\headwidth\vskip1pt	
\hrule\@height 0.5pt\@width\headwidth  			
\vskip-2\headrulewidth\vskip-1pt}      			
 \vspace{6mm}}     						
\makeatother

\graphicspath{{img/}}

\lstset{
 columns=fixed,
 basicstyle = \footnotesize,
 breakatwhitespace=false,         % 设置是否当且仅当在空白处自动中断.
 breaklines=true,
 numbers=left,                                        % 在左侧显示行号
 numberstyle=\tiny\color{gray},                       % 设定行号格式
 frame=none,                                          % 不显示背景边框
 backgroundcolor=\color[RGB]{245,245,244},            % 设定背景颜色
 keywordstyle=\color[RGB]{40,40,255},                 % 设定关键字颜色
 numberstyle=\footnotesize\color[RGB]{96,96,96},
 commentstyle=\color[RGB]{0,128,0},                % 设置代码注释的格式
 stringstyle=\rmfamily\slshape\color[RGB]{128,0,0},   % 设置字符串格式
 showstringspaces=false,                              % 不显示字符串中的空格
 language=JAVA,
 extendedchars=true,
 escapeinside=''                                       % 设置语言
}

%%%%%%%%%%%%%%%%%%%%%%%%%%%%%%%%%%%%%%%%%%%%%%
\numberwithin{equation}{section}
\newtheorem{myThm}{myThm}
\newtheorem*{myDef}{Definition}
\newtheorem*{mySol}{Solution}
\newtheorem*{myProof}{Proof}
\newcommand{\indep}{\rotatebox[origin=c]{90}{$\models$}}
\newcommand*\diff{\mathop{}\!\mathrm{d}}

\usepackage{multirow}
\renewcommand\refname{reference}
\author{组长:韩畅,组员:李展烁、王一之、闫旭芃}
\begin{document}
%\listoffigures
\captionsetup[figure]{labelfont={bf},labelformat={default},labelsep=period,name={图}}
\title{大数据综合处理实验\\
实验三}
\maketitle

\section{实验规划与设计}
\subsection{任务分配}
{171860551, \text{韩畅:组长}}\\ \indent
{171860550, \text{王一之:}}\\ \indent
{171860549, \text{闫旭芃:}}\\ \indent
{171840565, \text{李展烁:}}
\subsection{任务要求}
使用 MapReduce 完成对数据的二次排序。 
\subsection{设计思路}
map过程将每行的两个值处理为自定义数据结构Pair\_Int作为key,为减少传输量,value为空。
重写partitioner,保证分区时第一个值相同的值分配到同一个ruducer上。
在Pair\_Int中设置比较函数,每个分区会调用此比较函数,此时便完成了二次排序。
重载Comparator,保证分组时第一个值相同的结果在同一批次任务中处理,reduce阶段将传入的值转化为text直接输出即可。
\subsubsection{自定义数据类型}
自定义一个Pair\_Int存储数值对
\begin{figure}[H]
    \centering

    \includegraphics[width = 15cm]{pairint1.png}

    \caption{自定义Pair\_Int}
\end{figure}
重载读写方法,,重载toString方法方便最终输出,以二次排序的规则重载compareTo方法,同时重载WritableComparator提供字节流比较
\begin{figure}[H]
    \centering
    \subfigure[重载比较]
    {
        \includegraphics[width = 15cm]{pairint3.png}
    }
    \vfill
    \subfigure[重载比较2]
    {
        \includegraphics[width = 15cm]{pairint4.png}
    }
    \vfill
    \caption{Pair\_Int部分代码实现1}
    \label{PairInt}
\end{figure}
\begin{figure}[H]
    \centering
    \subfigure[重载读写]
    {
        \includegraphics[width = 15cm]{pairint2.png}
    }
    \vfill
    \subfigure[部分其他函数]
    {
        \includegraphics[width = 15cm]{pairint5.png}
    }

    \caption{Pair\_Int部分代码实现2}
\end{figure}
\subsubsection{主功能Map设计思路}
读入数据,每行的两个值以自定义数据结构Pair\_Int的方式作为key,value都为空
\begin{figure}[H]
    \centering

    \includegraphics[width = 15cm]{map1.png}

    \caption{Mapper实现}
    \label{mapper}
\end{figure}
\subsubsection{重写Partitioner}
重写Partitioner自定义分区,根据第一个值进行分区,保证第一个值相同在同一个reducer上
\begin{figure}[H]
    \centering

    \includegraphics[width = 15cm]{part1.png}

    \caption{Partitioner部分实现}
\end{figure}
\subsubsection{排序}
mapreduce此阶段对每个分区内进行排序,key为自定义类型,所以在Pair\_Int中自定义排序方法,见\ref{PairInt}\\
compareTo函数首先第一个值,若相同则比较第二个值,完成第一列升序,第二列降序的比较。

\subsubsection{自定义分组}
为保证结果正确,第一个值相同的结果要在同一个reduce任务中处理因此重载RawComparator进行自定义分组
\begin{figure}[H]
    \centering

    \includegraphics[width = 15cm]{comp1.png}

    \caption{重写Comparator}
\end{figure}
\subsubsection{主功能Reduce设计思路}
输入的key中已经是排好序的结果,自定义类型已经实现了tostring,直接输出即可
\begin{figure}[H]
    \centering

    \includegraphics[width = 15cm]{reduce1.png}

    \caption{重写Reducer}
    \label{reducer}
\end{figure}

\subsubsection{Key-Value类型协调}
map输出类型为 intPair, IntWritable,value实际为空(NullWritable) \\
reduce输出类型为 intPair, IntWritable,value同样为空(NullWritable),但key中会调用Pair\_Int的tostring,输出实际为text
\subsection{代码演示}
\subsubsection{Map阶段代码演示}
已经包含在设计思路图片中,见图\ref{mapper}

\subsubsection{Reduce阶段代码演示}
同上,见图\ref{reducer}
\section{实验结果展示}

\subsection{结果文件在HDFS上的路径}
todo
\subsection{输出结果文件的部分截图}
\begin{figure}[H]
	\centering
	\subfigure[结果开头]
	{
		\includegraphics[width = 15cm]{res1.PNG}
	}
	\vfill
	\subfigure[结果结尾]
	{
		\includegraphics[width = 15cm]{res2.PNG}
	}
	
	\caption{执行结果}
\end{figure}

\subsection{Web UI 报告内容展示}
\begin{figure}[H]
	\centering
	\includegraphics[width = 15cm]{job.PNG}
	\caption{WebUI执行报告}
\end{figure}
\section{实验经验总结与改进方向}
\begin{itemize}
	\item todo
	\item todo
\end{itemize}

\bibliographystyle{plain}
\bibliography{ref}

\end{document}
